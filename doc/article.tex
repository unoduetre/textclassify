\documentclass[a4paper]{classrep}
\usepackage{polski}
\usepackage{ucs}
\usepackage[utf8x]{inputenc}
\usepackage[T1]{fontenc}
\usepackage{cite}
%\usepackage{tcolorbox}
%\usepackage{apacite}
\usepackage{cancel}
\usepackage{graphicx}
\usepackage{dcolumn}
\usepackage{color}
\usepackage{colortbl}
\usepackage{enumerate}
\usepackage{url}
\usepackage[squaren]{SIunits}
\usepackage{icomma}
\usepackage{hyperref}
%\usepackage{url}
\usepackage{float}
\usepackage{indentfirst}
\usepackage{amssymb}
\usepackage{verbatim}
\usepackage{tabulary}
\usepackage{longtable}
\usepackage{rotating}
\usepackage{formular}
\usepackage{marginnote}
\usepackage{listingsutf8}
\usepackage{ifpdf}
%\usepackage{natbib}
\usepackage{fancybox}
\lstset{inputencoding=utf8/latin2,breaklines=true}
\newFRMfield{wyraz}{2cm}
\newif\ifShowAnswers
%\ShowAnswerstrue
\newcommand{\fillin}[2]{\ifShowAnswers{#1}\else{\marginnote{#2}\useFRMfield{wyraz}}\fi}
\newtheorem{xdefinicja}{Definicja}
\newenvironment{definicja}{\begin{xdefinicja}\normalfont}{\end{xdefinicja}}
\ifpdf
\newcommand{\obraz}[3]{
\begin{figure}[H]
\centering
\includegraphics[width=9cm]{#1.pdf}
\caption{#3}
\label{fig:#2}
\end{figure}
}
\newcommand{\malyobraz}[3]{
\begin{figure}[H]
\centering
\includegraphics[width=4cm]{#1.pdf}
\caption{#3}
\label{fig:#2}
\end{figure}
}
\newcommand{\podwojnyobraz}[4]{
\begin{figure}[H]
\centering
\includegraphics[width=4cm]{#1.pdf}
\includegraphics[width=4cm]{#2.pdf}
\caption{#4}
\label{fig:#3}
\end{figure}
}
\newcommand{\poziomyobraz}[2]{
\begin{sidewaysfigure}
\centering
\includegraphics[width=25cm]{#1.pdf}
%\caption{#2}
\label{fig:#2}
\end{sidewaysfigure}
}
\else
\newcommand{\obraz}[3]{
\begin{figure}[H]
\centering
\includegraphics[width=9cm]{#1.eps}
\caption{#3}
\label{fig:#2}
\end{figure}
}
\newcommand{\malyobraz}[3]{
\begin{figure}[H]
\centering
\includegraphics[width=4cm]{#1.eps}
\caption{#3}
\label{fig:#2}
\end{figure}
}
\newcommand{\podwojnyobraz}[4]{
\begin{figure}[H]
\centering
\includegraphics[width=4cm]{#1.eps}
\includegraphics[width=4cm]{#2.eps}
\caption{#4}
\label{fig:#3}
\end{figure}
}
\newcommand{\poziomyobraz}[2]{
\begin{sidewaysfigure}
\centering
\includegraphics[width=25cm]{#1.eps}
%\caption{#2}
\label{fig:#2}
\end{sidewaysfigure}
}
\fi
\newcommand{\rysunek}[4]{
\begin{figure}[H]
\centering
\begin{picture}#3
#4
\end{picture}
\caption{#2}
\label{fig:#1}
\end{figure}
}
\newcommand{\tabela}[4]{
\begin{table}[H]
\centering
\begin{tabular}[t]{#3}
#4
\end{tabular}
\caption{#2}
\label{tab:#1}
\end{table}
}

\newcommand{\tabelap}[4]{
\begin{table}[p]
\centering
\begin{tabular}[t]{#3}
#4
\end{tabular}
\caption{#2}
\label{tab:#1}
\end{table}
}

\newcommand{\tabelah}[4]{
\begin{table}[ht]
\centering
\begin{tabular}[t]{#3}
#4
\end{tabular}
\caption{#2}
\label{tab:#1}
\end{table}
}

\newcommand{\tabelat}[4]{
\begin{table}[t]
\centering
\begin{tabular}[t]{#3}
#4
\end{tabular}
\caption{#2}
\label{tab:#1}
\end{table}
}

\newcommand{\tabelab}[4]{
\begin{table}[b]
\centering
\begin{tabular}[t]{#3}
#4
\end{tabular}
\caption{#2}
\label{tab:#1}
\end{table}
}

\newcommand{\ltabela}[3]{
\clearpage
\begin{longtable}[c]{#2}
\caption{#1}\\
#3
\end{longtable}
}

\lstdefinelanguage{Smalltalk}{
  morekeywords={true,false,self,super,nil},
  sensitive=true,
  morecomment=[s]{"}{"},
  morestring=[d]',
}
\lstdefinestyle{SmalltalkStyle}{
  literate={:=}{{$\gets$}}1{^}{{$\uparrow$}}1
} 

\usepackage{amsfonts}
\usepackage[intlimits]{amsmath}

\usepackage{multirow}

\studycycle{Informatyka, studia dzienne, II st.}
\coursesemester{II}

\coursename{Komputerowe systemy rozpoznawania}
\courseyear{2013/2014}

\courseteacher{prof. dr hab. inż. Adam Niewiadomski}
\coursegroup{poniedziałek, 10:30}

\author{%
  \studentinfo{Mateusz Grotek}{186816} \and
  \studentinfo{Paweł Tarasiuk}{186875}
}

\title{Zadanie 1.: Ekstrakcja cech, miary podobieństwa, klasyfikacja}

\begin{document}
\maketitle
\section{Cel}
Celem niniejszego zadania jest zbadanie różnych metod ekstrakcji cech
oraz miar podobieństwa dla tekstu i zastosowanie ich w procesie klasyfikacji słabym klasyfikatorem kNN.
Omówione zostaną metody znane w literaturze oraz nasze własne pomysły,
a wszystkie te elementy znajdą swoje odzwierciedlenie w przygotowanej
przez nas implementacji, co umożliwi nam wygenerowanie i ocenę wyników
działania różnych metod.
\section{Wprowadzenie}
Klasyfikator kNN (k nearest neighbours, k najbliższych sąsiadów) to słaby klasyfikator, posiadający jednak istotną zaletę jaką jest prostota implementacji.
Dlatego jest on szeroko stosowany, gdyż pomimo swojej prostoty daje relatywnie dobre wyniki. Dodatkowo można go użyć w technikach typu AdaBoost,
które pozwalają z kilku słabych klasyfikatorów stworzyć jeden silny. Jego idea polega na wybraniu takiej kategorii która jest najczęstsza wśród sąsiadów wektora podlegającego klasyfikacji.
Potencjalni sąsiedzi są dostarczani do klasyfikatora na etapie jego uczenia. Aby wskazać najbliższych sąsiadów należy oczywiście użyć jakiejś miary odległości, czyli metryki.
Istnieje też odmiana klasyfikatora, którą moglibyśmy określić jako k najbardziej podobnych sąsiadów. W tej odmianie zamiast odległości w zadanej metryce używana jest funkcja podobieństwa.
Pokazuje to jak szerokie zastosowanie może mieć ten klasyfikator.

W naszym projekcie użyliśmy następujących funkcji podobieństwa:
\begin{itemize}
\item miara Jaccarda,
\item metoda \(n\)-gramów,
\item autorska metoda bazująca na słowach kluczowych
\end{itemize}
Użyte zostały następujące metryki:
\begin{itemize}
\item metryka euklidesowa
\item metryka uliczna
\item metryka Czebyszewa
\end{itemize}

Klasyfikacji można dokonywać bezpośrednio na podanych tekstach, jednak takie rozwiązanie charakteryzuje się wysokim kosztem obliczeniowym. Dlatego często lepiej jest
wyekstrachować ze zbiorów pewne wektory cech, które pozwolą szybko sklasyfikować dane teksty. Wybór wektorów cech zależy od postawionego zadania.
W naszym projekcie użyliśmy trzech sposobów ekstrakcji cech:
\begin{itemize}
\item naiwny sposób bazujący na użyciu wszystkich słów
\item bazujący na macierzy częstości terminów
\item autorska matoda bazująca na zbiorach rozmytych
\end{itemize}

Sama klasyfikacja została wykonana na dwóch zbiorach danych, każdy podzielony na podzbiór uczący i podzbiór testowy, przy czym sposób podziału jest wybierany w aplikacji.
Można dokonać podziału w sposób losowy (przy zadanych proporcjach), lub w ustalony sposób (60\% początkowych tekstów jako zbiór uczący, reszta to zbiór testowy).
Pierwszy zbiór danych, to zestaw krótkich wiadomości prasowych firmy Reuters. Drugi zestaw przygotowany przez nas, to teksty znanych pisarzy. Teksty prasowe zostały
sklasyfikowane na podstawie dwóch kategorii: miejsca którego dotyczą i tematu. Jeśli chodzi o teksty znanych pisarzy, to zostały one sklasyfikowane kategorią, którą są nazwiska pisarzy.

Poniżej omówimy autorskie metody ekstrakcji cech i miarę podobieństwa, których użyliśmy. 
Główną ideą naszej metody ekstrakcji są słowa kluczowe. Zbiór takich słów należy przygotować dla konkretnego zadania. Może on być wyznaczony ręcznie,
lub wygenerowany automatycznie na podstawie danych uczących. Zbiór takich słów dzielimy na podzbiory na podstawie ich znaczenia. Można oczywiście używać tylko jednego podzbioru
w którym są umieszczone wszystkie słowa kluczowe, ale zwiększenie ilości podzbiorów może poprawić klasyfikację, ze względu na zwiększenie wymiarowości przestrzeni. 
W jednym podzbiorze powinny być słowa oznaczająca podobne obiekty, na przykład nazwy geograficzne. Następnie dla każdej kategorii względem której klasyfikujemy przygotowujemy
osobny zestaw podzbiorów. Inaczej mówiąc jeżeli kategorią jest ,,kanada'' a jednym w jednym z podzbiorów znajdują się ,,organizacje'', to tworzymy podzbiór ,,kanadyjskie organizacje''.
Widać, że jest to metoda analogiczna do tego jak ludzie kategoryzują obiekty. Wystarczy następnie zauważyć, że określenie ,,kanadyjskie organizacji'' możemy modelować
jako pewien zbiór rozmyty. Niektóre organizacje są stricte kanadyjskie, inne tylko częściowo. Na podstawie danych uczących, lub wiedzy eksperckiej jesteśmy w stanie podać
funkcję przynależności do danego zbioru.

Mając taki zestaw zbiorów możemy wyekstrachować cechy z tekstów zbioru uczącego i na tej podstawie nauczyć klasyfikator. Ekstrakcja przebiega następująco. Wstępnie każdy element wektora
cech jest liczbą, która jest sumą wartości funkcji przynależności dla każdego słowa, kluczowego z podanego zbioru rozmytego, które wystąpiło w podanym tekście.
Opisuje to wzór
\[f_1(T,n)=\sum_{s\in K(n) \cap T} \mu_{K(n)}(s),\]
przy czym \(T\) to tekst z którego ekstrachujemy cechy, \(n\) to numer składowej wektora cech którą obliczamy, \(K(n)\) to \(n\)-ty zbiór rozmyty, \(\mu(s)\) to funkcja przynależności
dla zbioru \(K(n)\).
Następnie wektor taki jest normalizowany do długości równej \(1\).
Przy obliczaniu funkcji podobieństwa również użyliśmy słów kluczowych. Dla wybranych słów kluczowych liczony jest prosty współczynnik dopasowania (simple matching coefficient).
%Tworzymy 36 zbiorów rozmytych, po jednym dla każdej kombinacji podzbioru słów kluczowych i klasy.
%Do każdego zbioru rozmytego dla określonej kategorii i określonej klasy wrzucamy wszystkie słowa kluczowe z tej kategorii i początkowo przypisujemy im \(0{,}0\) jako wartośc
%funkcji przynależności do zbioru. Następnie przeglądamy zbiór uczący i do każdego elementu elementu przypisujemy stosunek pojawiania się danego słowa kluczowego w danej klasie
%względem jego pojawiania się we wszystkich klasach. Na przykład, jeżeli słowo pojawiło się w tekstach dotyczących kanady \(x\) razy, a we wszystkich tekstach pojawiło się \(y\) razy,
%to stosunek wynosi \(x/y\). Stosunek ten mówi jak dobrze słowo pasuje do podanej klasy. Na wejście kNN podajemy wektory uczące zawierające 36 wartości. Wektory te obliczamy dla
%tekstów uczących jako sumę wartości funkcji przynależności wystąpień każdego słowa podzieloną przez długość danego wektora.
%
%Algorytm ten jest rozszerzoną wersją prostszego algorytmu, który po prostu zlicza wystąpienia słów kluczowych w tekście. Dzięki uwzględnieniu rozmytości nie ma problemu ze słowami,
%które dobrze opisują więcej niż jedną klasę. Na przykład angielskie nazwiska pasują równie dobrze do wielkiej brytanii, jak i usa, więc będą miały mniejszą wagę,
%niż nazwiska japońskie przy rozpoznawaniu kraju.
%
%Dodatkową regułą, którą chcielibyśmy zastosować jest domyślne przypisanie do klasy usa w wypadku, gdy wartości dla innych klas nie przekraczają pewnej wartości. Reguła ta będzie
%szczególnie użyteczna w wypadku rozróżnienia między usa i uk.
\section{Opis implementacji}
%{\color{blue}
%Należy tu zamieścić krótki i zwięzły opis zaprojektowanych klas oraz powiązań
%między nimi. Powinien się tu również znaleźć diagram UML  (diagram klas)
%prezentujący najistotniejsze elementy stworzonej aplikacji. Należy także
%podać, w jakim języku programowania została stworzona aplikacja. }
Implementacja programu została wykonana w języku Java. Poniżej prezentujemy orientacyjny wycinek diagramu klas naszego programu.
\obraz{uml}{uml}{Wycinek uproszczonego diagramu klas aplikacji}
\section{Materiały i metody}
%{\color{blue}
%W tym miejscu należy opisać, jak przeprowadzone zostały wszystkie badania,
%których wyniki i dyskusja zamieszczane są w dalszych sekcjach. Opis ten
%powinien być na tyle dokładny, aby osoba czytająca go potrafiła wszystkie
%przeprowadzone badania samodzielnie powtórzyć w celu zweryfikowania ich
%poprawności (a zatem m.in. należy zamieścić tu opis architektury sieci,
%wartości współczynników użytych w kolejnych eksperymentach, sposób
%inicjalizacji wag, metodę uczenia itp. oraz informacje o danych, na których
%prowadzone były badania). Przy opisie należy odwoływać się i stosować do
%opisanych w sekcji drugiej wzorów i oznaczeń, a także w jasny sposób opisać
%cel konkretnego testu. Najlepiej byłoby wyraźnie wyszczególnić (ponumerować)
%poszczególne eksperymenty tak, aby łatwo było się do nich odwoływać dalej.}

Przygotowane rozwiązanie zawiera w sobie także interfejs graficzny wykonany
przy wykorzystaniu biblioteki Swing. Wybór ten został podyktowany tym, że
biblioteka ta jest domyślnie obecna w środowisku uruchomieniowym Javy,
dzięki czemu nasz projekt nie ma żadnych dodatkowych zewnętrznych zależności.
Interfejs pozwala wybrać przetwarzany zestaw danych, dopasować parametry
wskazywania zbioru uczącego i testowego oraz daje dostęp do wszystkich
dostępnych w projekcie metod i parametrów ekstrakcji cech, porównywania próbek
i klasyfikacji. Po ustaleniu wszystkich właściwości zadanego problemu
program wyświetla uogólnioną informację o tym, jaka część próbek została sklasyfikowana
poprawnie oraz pozwala zapisać szczegółowe wyniki w plikach csv. Dostępne są
trzy tryby generowania szczegółowych wyników:
\begin{itemize}
\item Raport z poszczególnych próbek ze zbioru testowego (do jakiej kategorii powinny należeć, a do jakiej zostały zaklasyfikowane)
\item Statystyki TPR (jaka część próbek które powinny do niej należeć faktycznie do niej trafiła) i PPV (jaka część próbek zaklasyfikowanych do danej kategorii faktycznie do niej należy) opisujące każdą z kategorii
\item Macierz, w której dla każdej pary właściwa kategoria/kategoria wskazana przez klasyfikator przechowywane są liczby przypadków w których tak się stało
\end{itemize}

{\color{blue}
W tym miejscu należy opisać, jak przeprowadzone zostały wszystkie badania,
których wyniki i dyskusja zamieszczane są w dalszych sekcjach. Opis ten
powinien być na tyle dokładny, aby osoba czytająca go potrafiła wszystkie
przeprowadzone badania samodzielnie powtórzyć w celu zweryfikowania ich
poprawności (a zatem m.in. należy zamieścić tu opis architektury sieci,
wartości współczynników użytych w kolejnych eksperymentach, sposób
inicjalizacji wag, metodę uczenia itp. oraz informacje o danych, na których
prowadzone były badania). Przy opisie należy odwoływać się i stosować do
opisanych w sekcji drugiej wzorów i oznaczeń, a także w jasny sposób opisać
cel konkretnego testu. Najlepiej byłoby wyraźnie wyszczególnić (ponumerować)
poszczególne eksperymenty tak, aby łatwo było się do nich odwoływać dalej.}

\section{Wyniki}
%{\color{blue}
%W tej sekcji należy zaprezentować, dla każdego przeprowadzonego eksperymentu,
%kompletny zestaw wyników w postaci tabel, wykresów itp. Powinny być one tak
%ponazywane, aby było wiadomo, do czego się odnoszą. Wszystkie tabele i wykresy
%należy oczywiście opisać (opisać co jest na osiach, w kolumnach itd.) stosując
%się do przyjętych wcześniej oznaczeń. Nie należy tu komentować i interpretować
%wyników, gdyż miejsce na to jest w kolejnej sekcji. Tu również dobrze jest
%wprowadzić oznaczenia (tabel, wykresów) aby móc się do nich odwoływać
%poniżej.}

Eksperymenty przeprowadzane były dla trzech problemów klasyfikacji:
\begin{itemize}
\item \textbf{Kraje} -- klasyfikacja próbek tekstowych ze zbioru
\emph{Reuters-21578 Text Categorization Collection Data Set}, gdzie jako etykiety
wybrane zostały kraje, których dotyczy dana próbka. Wybrane zostały wyłącznie
te próbki, które dotyczą dokładnie jednego kraju i w przypadku których krajem tym
jest Kanada, Francja, Japonia, Wielka Brytania, USA lub RFN. W przypadku
tego problemu mamy do dyspozycji \textbf{13441} próbek, w których (po uproszczonej
stemizacji i usunięciu słów nieznaczących) występuje \textbf{28145} różnych słów.
\item \textbf{Tematy} -- próbki wybierane z tego samego zbioru, co w przypadku
problemu \emph{Kraje}, jednakże tym razem klasami względem których chcemy rozróżniać
próbki są tematy artykułów. Wzięte pod uwagę zostały te próbki, których lista tematów
zawiera temat \emph{interest} albo temat \emph{grain} (próbki dotyczące obu tych
tematów jednocześnie były pomijane). Problem klasyfikacji polega zatem na podziale
zbioru wybranych próbek na dwie klasy, odpowiadające charakterystycznym tematom.
W przypadku tego problemu mamy do dyspozycji \textbf{998} próbek, w których
występuje \textbf{7533} rozróżnialnych, uwzględnianych przez nasz program słów.
\item \textbf{Autorzy} -- tym razem wykorzystaliśmy własny zbiór tekstów, przygotowany
w oparciu o dane przygotowane w ramach inicjatywy \emph{Project Gutenberg}. Z danych
tekstowych z plików \emph{The Complete Works of William Shakespeare} \cite{2} oraz
\emph{The Works of Lord Byron, Vol. 4} \cite{3} usunięte zostały przypisy, a następnie wybrane
zostały ciągłe próbki zawierające po kilka zdań lub wersów, tak aby każda z nich
zawierała od 50 do 200 słów. Kryterium klasyfikacji tak przygotowanych próbek
jest autor tekstu (po usunięciu przypisów, próbki wybierane są wprost z dzieł
literackich, więc autorem każdej z próbek jest William Shakespeare lub Lord George Gordon Byron).
W przypadku tego problemu mamy do dyspozycji \textbf{568} próbek, w których
występuje \textbf{6028} rozróżnialnych, uwzględnianych przez nasz program słów.
\end{itemize}

\vskip\baselineskip

Dla każdego ze przedstawionych powyżej problemów testowaliśmy różne podejścia do wyboru zbioru
uczącego oraz zbioru sprawdzającego. Domyślnym ustawieniem było potraktowanie $60\%$ początkowych
próbek jako zbioru uczącego, zaś pozostałych $40\%$ próbek -- jako zbioru sprawdzającego.
Oceniane było także zachowanie na mniejszych zbiorach -- wtedy zbiór uczący wybierany był
zawsze od początku, zaś zbiór sprawdzający -- od końca (podanie odpowiedniego argumentu
w przygotowanej przez nas implementacji powoduje, że dozwolone jest nachodzenie na siebie
tych dwóch zbiorów, zatem suma ich miar może przekroczyć liczbę wszystkich próbek w danym
problemie). Dodatkowa opcja pozwala spowodować, aby  przy części testów wybierać próbki
w sposób pseudolosowy (zamiast stosowania schematu opisanego powyżej).

{

\makeatletter
\newcommand\nobreakcline[1]{\@nobreakcline#1\@nil}%
\def\@nobreakcline#1-#2\@nil{%
  \omit
  \@multicnt#1%
  \advance\@multispan\m@ne
  \ifnum\@multicnt=\@ne\@firstofone{&\omit}\fi
  \@multicnt#2%
  \advance\@multicnt-#1%
  \advance\@multispan\@ne
  \leaders\hrule\@height\arrayrulewidth\hfill\\*
  \noalign{\vskip-\arrayrulewidth}}
\makeatother

\begingroup
\setlength{\LTleft}{-20cm plus -1fill}
\setlength{\LTright}{\LTleft}
\ltabela{Wyniki przeprowadzonych prób}{|c|c|c|c|c|c|c|c|}{ \hline
\textbf{\#} & \textbf{Zestaw} & \textbf{Zb. tr./ucz.} & \textbf{Podobieństwo} & $\mathbf{k}$ & \textbf{Kategoria} & \textbf{TPR} & \textbf{PPV} \\
\hline
\endhead
\multirow{7}{*}{1} & \multirow{7}{*}{Kraje} & \multirow{7}{*}{$40\%$/$60\%$} & \multirow{7}{*}{$R^n$, wybr. sł., $d_e$} & \multirow{7}{*}{5} & canada & $28,39\%$ & $72,00\%$ \\*
 & & & & & france & $56,18\%$ & $75,76\%$ \\*
 & & & & & japan & $65,09\%$ & $82,63\%$ \\*
 & & & & & uk & $65,69\%$ & $84,81\%$ \\*
 & & & & & usa & $97,71\%$ & $89,71\%$ \\*
 & & & & & west-germany & $41,04\%$ & $70,51\%$ \\*
\nobreakcline{6-8}
& & & & & \textbf{poprawne:} & \multicolumn{2}{c|}{$\mathbf{88,41\%}$} \\
\hline
\multirow{7}{*}{2} & \multirow{7}{*}{Kraje} & \multirow{7}{*}{$40\%$/$60\%$} & \multirow{7}{*}{$R^n$, zb. rozm., $d_e$} & \multirow{7}{*}{5} & canada & $28,08\%$ & $72,36\%$ \\*
 & & & & & france & $60,67\%$ & $68,35\%$ \\*
 & & & & & japan & $74,53\%$ & $82,29\%$ \\*
 & & & & & uk & $68,63\%$ & $80,15\%$ \\*
 & & & & & usa & $97,62\%$ & $90,59\%$ \\*
 & & & & & west-germany & $38,81\%$ & $77,61\%$ \\*
\nobreakcline{6-8}
& & & & & \textbf{poprawne:} & \multicolumn{2}{c|}{$\mathbf{88,88\%}$} \\
\hline
\multirow{7}{*}{3} & \multirow{7}{*}{Kraje} & \multirow{7}{*}{$40\%$/$60\%$} & \multirow{7}{*}{$R^n$, zb. rozm., $d_\infty$} & \multirow{7}{*}{5} & canada & $29,34\%$ & $72,09\%$ \\*
 & & & & & france & $66,29\%$ & $68,60\%$ \\*
 & & & & & japan & $75,94\%$ & $83,85\%$ \\*
 & & & & & uk & $66,99\%$ & $79,77\%$ \\*
 & & & & & usa & $97,52\%$ & $90,82\%$ \\*
 & & & & & west-germany & $41,79\%$ & $74,67\%$ \\*
\nobreakcline{6-8}
& & & & & \textbf{poprawne:} & \multicolumn{2}{c|}{$\mathbf{89,01\%}$} \\
\hline
\multirow{7}{*}{4} & \multirow{7}{*}{Kraje} & \multirow{7}{*}{$40\%$/$60\%$} & \multirow{7}{*}{$R^n$, zb. rozm., $d_1$} & \multirow{7}{*}{5} & canada & $28,08\%$ & $72,36\%$ \\*
 & & & & & france & $53,93\%$ & $72,73\%$ \\*
 & & & & & japan & $70,75\%$ & $82,87\%$ \\*
 & & & & & uk & $65,69\%$ & $83,06\%$ \\*
 & & & & & usa & $97,89\%$ & $90,00\%$ \\*
 & & & & & west-germany & $38,81\%$ & $77,61\%$ \\*
\nobreakcline{6-8}
& & & & & \textbf{poprawne:} & \multicolumn{2}{c|}{$\mathbf{88,67\%}$} \\
\hline
\multirow{7}{*}{5} & \multirow{7}{*}{Kraje} & \multirow{7}{*}{$40\%$/$60\%$} & \multirow{7}{*}{$R^n$, zb. rozm., $d_e$} & \multirow{7}{*}{3} & canada & $31,55\%$ & $38,76\%$ \\*
 & & & & & france & $58,43\%$ & $57,78\%$ \\*
 & & & & & japan & $72,17\%$ & $80,95\%$ \\*
 & & & & & uk & $65,36\%$ & $80,32\%$ \\*
 & & & & & usa & $94,49\%$ & $90,55\%$ \\*
 & & & & & west-germany & $43,28\%$ & $69,05\%$ \\*
\nobreakcline{6-8}
& & & & & \textbf{poprawne:} & \multicolumn{2}{c|}{$\mathbf{86,37\%}$} \\
\hline
\multirow{7}{*}{6} & \multirow{7}{*}{Kraje} & \multirow{7}{*}{$40\%$/$60\%$} & \multirow{7}{*}{$R^n$, zb. rozm., $d_e$} & \multirow{7}{*}{11} & canada & $26,50\%$ & $75,68\%$ \\*
 & & & & & france & $53,93\%$ & $71,64\%$ \\*
 & & & & & japan & $73,58\%$ & $80,41\%$ \\*
 & & & & & uk & $65,69\%$ & $83,40\%$ \\*
 & & & & & usa & $97,85\%$ & $89,90\%$ \\*
 & & & & & west-germany & $35,82\%$ & $76,19\%$ \\*
\nobreakcline{6-8}
& & & & & \textbf{poprawne:} & \multicolumn{2}{c|}{$\mathbf{88,58\%}$} \\
\hline
\multirow{7}{*}{7} & \multirow{7}{*}{Kraje} & \multirow{7}{*}{$20\%$/$30\%$, losowanie} & \multirow{7}{*}{$R^n$, zb. rozm., $d_e$} & \multirow{7}{*}{5} & canada & $34,87\%$ & $73,61\%$ \\*
 & & & & & france & $50,00\%$ & $48,39\%$ \\*
 & & & & & japan & $65,66\%$ & $80,25\%$ \\*
 & & & & & uk & $60,71\%$ & $78,16\%$ \\*
 & & & & & usa & $97,46\%$ & $91,04\%$ \\*
 & & & & & west-germany & $18,52\%$ & $55,56\%$ \\*
\nobreakcline{6-8}
& & & & & \textbf{poprawne:} & \multicolumn{2}{c|}{$\mathbf{89,10\%}$} \\
\hline
\multirow{7}{*}{8} & \multirow{7}{*}{Kraje} & \multirow{7}{*}{$40\%$/$60\%$, losowanie} & \multirow{7}{*}{$R^n$, zb. rozm., $d_e$} & \multirow{7}{*}{5} & canada & $29,57\%$ & $72,04\%$ \\*
 & & & & & france & $60,14\%$ & $70,34\%$ \\*
 & & & & & japan & $67,97\%$ & $83,77\%$ \\*
 & & & & & uk & $70,43\%$ & $83,26\%$ \\*
 & & & & & usa & $97,19\%$ & $89,39\%$ \\*
 & & & & & west-germany & $34,36\%$ & $67,00\%$ \\*
\nobreakcline{6-8}
& & & & & \textbf{poprawne:} & \multicolumn{2}{c|}{$\mathbf{87,86\%}$} \\
\hline
\multirow{7}{*}{9} & \multirow{7}{*}{Kraje} & \multirow{7}{*}{$40\%$/$60\%$, losowanie, zachodzenie} & \multirow{7}{*}{$R^n$, zb. rozm., $d_e$} & \multirow{7}{*}{5} & canada & $32,07\%$ & $69,70\%$ \\*
 & & & & & france & $66,01\%$ & $78,91\%$ \\*
 & & & & & japan & $75,09\%$ & $83,14\%$ \\*
 & & & & & uk & $74,32\%$ & $85,36\%$ \\*
 & & & & & usa & $97,32\%$ & $90,75\%$ \\*
 & & & & & west-germany & $52,26\%$ & $78,20\%$ \\*
\nobreakcline{6-8}
& & & & & \textbf{poprawne:} & \multicolumn{2}{c|}{$\mathbf{89,19\%}$} \\
\hline
\multirow{7}{*}{10} & \multirow{7}{*}{Kraje} & \multirow{7}{*}{$60\%$/$90\%$, losowanie, zachodzenie} & \multirow{7}{*}{$R^n$, zb. rozm., $d_e$} & \multirow{7}{*}{5} & canada & $33,99\%$ & $78,25\%$ \\*
 & & & & & france & $71,24\%$ & $82,56\%$ \\*
 & & & & & japan & $79,19\%$ & $88,50\%$ \\*
 & & & & & uk & $74,33\%$ & $87,91\%$ \\*
 & & & & & usa & $98,20\%$ & $91,22\%$ \\*
 & & & & & west-germany & $45,67\%$ & $82,53\%$ \\*
\nobreakcline{6-8}
& & & & & \textbf{poprawne:} & \multicolumn{2}{c|}{$\mathbf{90,35\%}$} \\
\hline
\multirow{7}{*}{11} & \multirow{7}{*}{Kraje} & \multirow{7}{*}{$20\%$/$30\%$} & \multirow{7}{*}{$R^n$, wszystko, $d_e$} & \multirow{7}{*}{5} & canada & $5,26\%$ & $16,33\%$ \\*
 & & & & & france & $36,67\%$ & $40,74\%$ \\*
 & & & & & japan & $39,39\%$ & $58,21\%$ \\*
 & & & & & uk & $61,61\%$ & $51,49\%$ \\*
 & & & & & usa & $95,94\%$ & $89,44\%$ \\*
 & & & & & west-germany & $12,96\%$ & $100,00\%$ \\*
\nobreakcline{6-8}
& & & & & \textbf{poprawne:} & \multicolumn{2}{c|}{$\mathbf{84,98\%}$} \\
\hline
\multirow{7}{*}{12} & \multirow{7}{*}{Kraje} & \multirow{7}{*}{$20\%$/$30\%$} & \multirow{7}{*}{$R^n$, wybr. sł., $d_e$} & \multirow{7}{*}{5} & canada & $32,89\%$ & $83,33\%$ \\*
 & & & & & france & $40,00\%$ & $50,00\%$ \\*
 & & & & & japan & $52,53\%$ & $75,36\%$ \\*
 & & & & & uk & $53,57\%$ & $69,77\%$ \\*
 & & & & & usa & $97,59\%$ & $90,38\%$ \\*
 & & & & & west-germany & $22,22\%$ & $41,38\%$ \\*
\nobreakcline{6-8}
& & & & & \textbf{poprawne:} & \multicolumn{2}{c|}{$\mathbf{88,29\%}$} \\
\hline
\multirow{7}{*}{13} & \multirow{7}{*}{Kraje} & \multirow{7}{*}{$20\%$/$30\%$} & \multirow{7}{*}{$R^n$, zb. rozm., $d_e$} & \multirow{7}{*}{5} & canada & $35,53\%$ & $72,97\%$ \\*
 & & & & & france & $50,00\%$ & $48,39\%$ \\*
 & & & & & japan & $64,65\%$ & $80,00\%$ \\*
 & & & & & uk & $60,71\%$ & $78,16\%$ \\*
 & & & & & usa & $97,32\%$ & $91,14\%$ \\*
 & & & & & west-germany & $24,07\%$ & $56,52\%$ \\*
\nobreakcline{6-8}
& & & & & \textbf{poprawne:} & \multicolumn{2}{c|}{$\mathbf{89,10\%}$} \\
\hline
\multirow{7}{*}{14} & \multirow{7}{*}{Kraje} & \multirow{7}{*}{$20\%$/$30\%$} & \multirow{7}{*}{Jaccard} & \multirow{7}{*}{5} & canada & $17,11\%$ & $50,00\%$ \\*
 & & & & & france & $40,00\%$ & $60,00\%$ \\*
 & & & & & japan & $54,55\%$ & $71,05\%$ \\*
 & & & & & uk & $67,86\%$ & $78,35\%$ \\*
 & & & & & usa & $98,39\%$ & $90,74\%$ \\*
 & & & & & west-germany & $20,37\%$ & $84,62\%$ \\*
\nobreakcline{6-8}
& & & & & \textbf{poprawne:} & \multicolumn{2}{c|}{$\mathbf{88,69\%}$} \\
\hline
\multirow{7}{*}{15} & \multirow{7}{*}{Kraje} & \multirow{7}{*}{$10\%$/$15\%$} & \multirow{7}{*}{$R^n$, wszystko, $d_e$} & \multirow{7}{*}{5} & canada & $3,28\%$ & $4,88\%$ \\*
 & & & & & france & $12,50\%$ & $9,09\%$ \\*
 & & & & & japan & $11,76\%$ & $80,00\%$ \\*
 & & & & & uk & $43,75\%$ & $26,92\%$ \\*
 & & & & & usa & $95,69\%$ & $91,58\%$ \\*
 & & & & & west-germany & $0,00\%$ & $0,00\%$ \\*
\nobreakcline{6-8}
& & & & & \textbf{poprawne:} & \multicolumn{2}{c|}{$\mathbf{85,65\%}$} \\
\hline
\multirow{7}{*}{16} & \multirow{7}{*}{Kraje} & \multirow{7}{*}{$10\%$/$15\%$} & \multirow{7}{*}{$R^n$, wybr. sł., $d_e$} & \multirow{7}{*}{5} & canada & $31,15\%$ & $90,48\%$ \\*
 & & & & & france & $25,00\%$ & $25,00\%$ \\*
 & & & & & japan & $52,94\%$ & $94,74\%$ \\*
 & & & & & uk & $34,38\%$ & $68,75\%$ \\*
 & & & & & usa & $98,56\%$ & $92,09\%$ \\*
 & & & & & west-germany & $21,43\%$ & $37,50\%$ \\*
\nobreakcline{6-8}
& & & & & \textbf{poprawne:} & \multicolumn{2}{c|}{$\mathbf{90,78\%}$} \\
\hline
\multirow{7}{*}{17} & \multirow{7}{*}{Kraje} & \multirow{7}{*}{$10\%$/$15\%$} & \multirow{7}{*}{$R^n$, zb. rozm., $d_e$} & \multirow{7}{*}{5} & canada & $29,51\%$ & $90,00\%$ \\*
 & & & & & france & $50,00\%$ & $57,14\%$ \\*
 & & & & & japan & $61,76\%$ & $84,00\%$ \\*
 & & & & & uk & $40,62\%$ & $72,22\%$ \\*
 & & & & & usa & $98,82\%$ & $92,33\%$ \\*
 & & & & & west-germany & $25,00\%$ & $70,00\%$ \\*
\nobreakcline{6-8}
& & & & & \textbf{poprawne:} & \multicolumn{2}{c|}{$\mathbf{91,52\%}$} \\
\hline
\multirow{7}{*}{18} & \multirow{7}{*}{Kraje} & \multirow{7}{*}{$10\%$/$15\%$} & \multirow{7}{*}{Jaccard} & \multirow{7}{*}{5} & canada & $16,39\%$ & $43,48\%$ \\*
 & & & & & france & $12,50\%$ & $33,33\%$ \\*
 & & & & & japan & $32,35\%$ & $91,67\%$ \\*
 & & & & & uk & $46,88\%$ & $55,56\%$ \\*
 & & & & & usa & $98,56\%$ & $91,23\%$ \\*
 & & & & & west-germany & $3,57\%$ & $33,33\%$ \\*
\nobreakcline{6-8}
& & & & & \textbf{poprawne:} & \multicolumn{2}{c|}{$\mathbf{89,44\%}$} \\
\hline
\multirow{7}{*}{19} & \multirow{7}{*}{Kraje} & \multirow{7}{*}{$10\%$/$15\%$} & \multirow{7}{*}{Wg. sł. kluczowych} & \multirow{7}{*}{5} & canada & $19,67\%$ & $92,31\%$ \\*
 & & & & & france & $0,00\%$ & $0,00\%$ \\*
 & & & & & japan & $41,18\%$ & $93,33\%$ \\*
 & & & & & uk & $31,25\%$ & $55,56\%$ \\*
 & & & & & usa & $99,24\%$ & $90,93\%$ \\*
 & & & & & west-germany & $10,71\%$ & $42,86\%$ \\*
\nobreakcline{6-8}
& & & & & \textbf{poprawne:} & \multicolumn{2}{c|}{$\mathbf{90,11\%}$} \\
\hline
\multirow{7}{*}{20} & \multirow{7}{*}{Kraje} & \multirow{7}{*}{$4\%$/$6\%$} & \multirow{7}{*}{$R^n$, wszystko, $d_e$} & \multirow{7}{*}{5} & canada & $0,00\%$ & $0,00\%$ \\*
 & & & & & france & $0,00\%$ & $0,00\%$ \\*
 & & & & & japan & $30,00\%$ & $50,00\%$ \\*
 & & & & & uk & $20,00\%$ & $33,33\%$ \\*
 & & & & & usa & $95,48\%$ & $92,26\%$ \\*
 & & & & & west-germany & $0,00\%$ & $0,00\%$ \\*
\nobreakcline{6-8}
& & & & & \textbf{poprawne:} & \multicolumn{2}{c|}{$\mathbf{87,36\%}$} \\
\hline
\multirow{7}{*}{21} & \multirow{7}{*}{Kraje} & \multirow{7}{*}{$4\%$/$6\%$} & \multirow{7}{*}{$R^n$, wszystko, $d_\infty$} & \multirow{7}{*}{5} & canada & $5,00\%$ & $5,88\%$ \\*
 & & & & & france & $0,00\%$ & $0,00\%$ \\*
 & & & & & japan & $40,00\%$ & $50,00\%$ \\*
 & & & & & uk & $10,00\%$ & $11,11\%$ \\*
 & & & & & usa & $94,25\%$ & $91,62\%$ \\*
 & & & & & west-germany & $0,00\%$ & $0,00\%$ \\*
\nobreakcline{6-8}
& & & & & \textbf{poprawne:} & \multicolumn{2}{c|}{$\mathbf{86,43\%}$} \\
\hline
\multirow{7}{*}{22} & \multirow{7}{*}{Kraje} & \multirow{7}{*}{$4\%$/$6\%$} & \multirow{7}{*}{$R^n$, wszystko, $d_1$} & \multirow{7}{*}{5} & canada & $0,00\%$ & $0,00\%$ \\*
 & & & & & france & $0,00\%$ & $0,00\%$ \\*
 & & & & & japan & $0,00\%$ & $0,00\%$ \\*
 & & & & & uk & $40,00\%$ & $11,76\%$ \\*
 & & & & & usa & $94,87\%$ & $92,77\%$ \\*
 & & & & & west-germany & $0,00\%$ & $0,00\%$ \\*
\nobreakcline{6-8}
& & & & & \textbf{poprawne:} & \multicolumn{2}{c|}{$\mathbf{86,62\%}$} \\
\hline
\multirow{7}{*}{23} & \multirow{7}{*}{Kraje} & \multirow{7}{*}{$4\%$/$6\%$} & \multirow{7}{*}{$R^n$, wszystko, $d_e$} & \multirow{7}{*}{3} & canada & $15,00\%$ & $8,57\%$ \\*
 & & & & & france & $0,00\%$ & $0,00\%$ \\*
 & & & & & japan & $30,00\%$ & $60,00\%$ \\*
 & & & & & uk & $20,00\%$ & $33,33\%$ \\*
 & & & & & usa & $92,20\%$ & $93,74\%$ \\*
 & & & & & west-germany & $0,00\%$ & $0,00\%$ \\*
\nobreakcline{6-8}
& & & & & \textbf{poprawne:} & \multicolumn{2}{c|}{$\mathbf{84,94\%}$} \\
\hline
\multirow{7}{*}{24} & \multirow{7}{*}{Kraje} & \multirow{7}{*}{$4\%$/$6\%$} & \multirow{7}{*}{$R^n$, wszystko, $d_e$} & \multirow{7}{*}{11} & canada & $0,00\%$ & $0,00\%$ \\*
 & & & & & france & $0,00\%$ & $0,00\%$ \\*
 & & & & & japan & $30,00\%$ & $60,00\%$ \\*
 & & & & & uk & $10,00\%$ & $33,33\%$ \\*
 & & & & & usa & $99,59\%$ & $91,68\%$ \\*
 & & & & & west-germany & $0,00\%$ & $0,00\%$ \\*
\nobreakcline{6-8}
& & & & & \textbf{poprawne:} & \multicolumn{2}{c|}{$\mathbf{90,89\%}$} \\
\hline
\multirow{7}{*}{25} & \multirow{7}{*}{Kraje} & \multirow{7}{*}{$4\%$/$6\%$} & \multirow{7}{*}{$R^n$, wybr. sł., $d_e$} & \multirow{7}{*}{5} & canada & $15,00\%$ & $100,00\%$ \\*
 & & & & & france & $33,33\%$ & $50,00\%$ \\*
 & & & & & japan & $40,00\%$ & $66,67\%$ \\*
 & & & & & uk & $20,00\%$ & $33,33\%$ \\*
 & & & & & usa & $98,97\%$ & $92,51\%$ \\*
 & & & & & west-germany & $0,00\%$ & $0,00\%$ \\*
\nobreakcline{6-8}
& & & & & \textbf{poprawne:} & \multicolumn{2}{c|}{$\mathbf{91,45\%}$} \\
\hline
\multirow{7}{*}{26} & \multirow{7}{*}{Kraje} & \multirow{7}{*}{$4\%$/$6\%$} & \multirow{7}{*}{$R^n$, zb. rozm., $d_e$} & \multirow{7}{*}{5} & canada & $25,00\%$ & $83,33\%$ \\*
 & & & & & france & $66,67\%$ & $66,67\%$ \\*
 & & & & & japan & $70,00\%$ & $100,00\%$ \\*
 & & & & & uk & $30,00\%$ & $60,00\%$ \\*
 & & & & & usa & $99,38\%$ & $93,80\%$ \\*
 & & & & & west-germany & $0,00\%$ & $0,00\%$ \\*
\nobreakcline{6-8}
& & & & & \textbf{poprawne:} & \multicolumn{2}{c|}{$\mathbf{93,12\%}$} \\
\hline
\multirow{7}{*}{27} & \multirow{7}{*}{Kraje} & \multirow{7}{*}{$4\%$/$6\%$} & \multirow{7}{*}{Jaccard} & \multirow{7}{*}{5} & canada & $5,00\%$ & $8,33\%$ \\*
 & & & & & france & $100,00\%$ & $100,00\%$ \\*
 & & & & & japan & $30,00\%$ & $50,00\%$ \\*
 & & & & & uk & $40,00\%$ & $44,44\%$ \\*
 & & & & & usa & $96,71\%$ & $92,72\%$ \\*
 & & & & & west-germany & $0,00\%$ & $0,00\%$ \\*
\nobreakcline{6-8}
& & & & & \textbf{poprawne:} & \multicolumn{2}{c|}{$\mathbf{89,59\%}$} \\
\hline
\multirow{7}{*}{28} & \multirow{7}{*}{Kraje} & \multirow{7}{*}{$4\%$/$6\%$} & \multirow{7}{*}{Wg. sł. kluczowych} & \multirow{7}{*}{5} & canada & $5,00\%$ & $50,00\%$ \\*
 & & & & & france & $0,00\%$ & $0,00\%$ \\*
 & & & & & japan & $10,00\%$ & $100,00\%$ \\*
 & & & & & uk & $20,00\%$ & $100,00\%$ \\*
 & & & & & usa & $99,79\%$ & $91,18\%$ \\*
 & & & & & west-germany & $0,00\%$ & $0,00\%$ \\*
\nobreakcline{6-8}
& & & & & \textbf{poprawne:} & \multicolumn{2}{c|}{$\mathbf{91,08\%}$} \\
\hline
\multirow{7}{*}{29} & \multirow{7}{*}{Kraje} & \multirow{7}{*}{$4\%$/$6\%$} & \multirow{7}{*}{fuzzy} & \multirow{7}{*}{5} & canada & $0,00\%$ & $0,00\%$ \\*
 & & & & & france & $0,00\%$ & $0,00\%$ \\*
 & & & & & japan & $0,00\%$ & $0,00\%$ \\*
 & & & & & uk & $10,00\%$ & $16,67\%$ \\*
 & & & & & usa & $98,36\%$ & $91,24\%$ \\*
 & & & & & west-germany & $0,00\%$ & $0,00\%$ \\*
\nobreakcline{6-8}
& & & & & \textbf{poprawne:} & \multicolumn{2}{c|}{$\mathbf{89,22\%}$} \\
\hline
\multirow{3}{*}{30} & \multirow{3}{*}{Tematy} & \multirow{3}{*}{$40\%$/$60\%$} & \multirow{3}{*}{$R^n$, wszystko, $d_e$} & \multirow{3}{*}{5} & grain & $93,78\%$ & $98,60\%$ \\*
 & & & & & interest & $98,29\%$ & $92,47\%$ \\*
\nobreakcline{6-8}
& & & & & \textbf{poprawne:} & \multicolumn{2}{c|}{$\mathbf{95,75\%}$} \\
\hline
\multirow{3}{*}{31} & \multirow{3}{*}{Tematy} & \multirow{3}{*}{$40\%$/$60\%$} & \multirow{3}{*}{$R^n$, wszystko, $d_\infty$} & \multirow{3}{*}{5} & grain & $84,89\%$ & $90,95\%$ \\*
 & & & & & interest & $89,14\%$ & $82,11\%$ \\*
\nobreakcline{6-8}
& & & & & \textbf{poprawne:} & \multicolumn{2}{c|}{$\mathbf{86,75\%}$} \\
\hline
\multirow{3}{*}{32} & \multirow{3}{*}{Tematy} & \multirow{3}{*}{$40\%$/$60\%$} & \multirow{3}{*}{$R^n$, wszystko, $d_1$} & \multirow{3}{*}{5} & grain & $83,56\%$ & $98,95\%$ \\*
 & & & & & interest & $98,86\%$ & $82,38\%$ \\*
\nobreakcline{6-8}
& & & & & \textbf{poprawne:} & \multicolumn{2}{c|}{$\mathbf{90,25\%}$} \\
\hline
\multirow{3}{*}{33} & \multirow{3}{*}{Tematy} & \multirow{3}{*}{$40\%$/$60\%$} & \multirow{3}{*}{$R^n$, wszystko, $d_e$} & \multirow{3}{*}{3} & grain & $93,78\%$ & $99,06\%$ \\*
 & & & & & interest & $98,86\%$ & $92,51\%$ \\*
\nobreakcline{6-8}
& & & & & \textbf{poprawne:} & \multicolumn{2}{c|}{$\mathbf{96,00\%}$} \\
\hline
\multirow{3}{*}{34} & \multirow{3}{*}{Tematy} & \multirow{3}{*}{$40\%$/$60\%$} & \multirow{3}{*}{$R^n$, wszystko, $d_e$} & \multirow{3}{*}{11} & grain & $94,67\%$ & $100,00\%$ \\*
 & & & & & interest & $100,00\%$ & $93,58\%$ \\*
\nobreakcline{6-8}
& & & & & \textbf{poprawne:} & \multicolumn{2}{c|}{$\mathbf{97,00\%}$} \\
\hline
\multirow{3}{*}{35} & \multirow{3}{*}{Tematy} & \multirow{3}{*}{$20\%$/$30\%$, losowanie} & \multirow{3}{*}{$R^n$, wszystko, $d_e$} & \multirow{3}{*}{5} & grain & $75,24\%$ & $100,00\%$ \\*
 & & & & & interest & $100,00\%$ & $78,51\%$ \\*
\nobreakcline{6-8}
& & & & & \textbf{poprawne:} & \multicolumn{2}{c|}{$\mathbf{87,00\%}$} \\
\hline
\multirow{3}{*}{36} & \multirow{3}{*}{Tematy} & \multirow{3}{*}{$40\%$/$60\%$, losowanie} & \multirow{3}{*}{$R^n$, wszystko, $d_e$} & \multirow{3}{*}{5} & grain & $86,03\%$ & $100,00\%$ \\*
 & & & & & interest & $100,00\%$ & $82,76\%$ \\*
\nobreakcline{6-8}
& & & & & \textbf{poprawne:} & \multicolumn{2}{c|}{$\mathbf{91,64\%}$} \\
\hline
\multirow{3}{*}{37} & \multirow{3}{*}{Tematy} & \multirow{3}{*}{$40\%$/$60\%$, losowanie, zachodzenie} & \multirow{3}{*}{$R^n$, wszystko, $d_e$} & \multirow{3}{*}{5} & grain & $91,81\%$ & $100,00\%$ \\*
 & & & & & interest & $100,00\%$ & $90,18\%$ \\*
\nobreakcline{6-8}
& & & & & \textbf{poprawne:} & \multicolumn{2}{c|}{$\mathbf{95,33\%}$} \\
\hline
\multirow{3}{*}{38} & \multirow{3}{*}{Tematy} & \multirow{3}{*}{$60\%$/$90\%$, losowanie, zachodzenie} & \multirow{3}{*}{$R^n$, wszystko, $d_e$} & \multirow{3}{*}{5} & grain & $92,04\%$ & $99,79\%$ \\*
 & & & & & interest & $99,74\%$ & $90,33\%$ \\*
\nobreakcline{6-8}
& & & & & \textbf{poprawne:} & \multicolumn{2}{c|}{$\mathbf{95,33\%}$} \\
\hline
\multirow{3}{*}{39} & \multirow{3}{*}{Tematy} & \multirow{3}{*}{$40\%$/$60\%$} & \multirow{3}{*}{$R^n$, wybr. sł., $d_e$} & \multirow{3}{*}{5} & grain & $85,78\%$ & $98,47\%$ \\*
 & & & & & interest & $98,29\%$ & $84,31\%$ \\*
\nobreakcline{6-8}
& & & & & \textbf{poprawne:} & \multicolumn{2}{c|}{$\mathbf{91,25\%}$} \\
\hline
\multirow{3}{*}{40} & \multirow{3}{*}{Tematy} & \multirow{3}{*}{$40\%$/$60\%$} & \multirow{3}{*}{$R^n$, zb. rozm., $d_e$} & \multirow{3}{*}{5} & grain & $98,67\%$ & $96,52\%$ \\*
 & & & & & interest & $95,43\%$ & $98,24\%$ \\*
\nobreakcline{6-8}
& & & & & \textbf{poprawne:} & \multicolumn{2}{c|}{$\mathbf{97,25\%}$} \\
\hline
\multirow{3}{*}{41} & \multirow{3}{*}{Tematy} & \multirow{3}{*}{$40\%$/$60\%$} & \multirow{3}{*}{Jaccard} & \multirow{3}{*}{5} & grain & $98,22\%$ & $99,10\%$ \\*
 & & & & & interest & $98,86\%$ & $97,74\%$ \\*
\nobreakcline{6-8}
& & & & & \textbf{poprawne:} & \multicolumn{2}{c|}{$\mathbf{98,50\%}$} \\
\hline
\multirow{3}{*}{42} & \multirow{3}{*}{Tematy} & \multirow{3}{*}{$40\%$/$60\%$} & \multirow{3}{*}{Wg. sł. kluczowych} & \multirow{3}{*}{5} & grain & $86,22\%$ & $100,00\%$ \\*
 & & & & & interest & $100,00\%$ & $84,95\%$ \\*
\nobreakcline{6-8}
& & & & & \textbf{poprawne:} & \multicolumn{2}{c|}{$\mathbf{92,25\%}$} \\
\hline
\multirow{3}{*}{43} & \multirow{3}{*}{Tematy} & \multirow{3}{*}{$40\%$/$60\%$} & \multirow{3}{*}{fuzzy} & \multirow{3}{*}{5} & grain & $93,33\%$ & $99,53\%$ \\*
 & & & & & interest & $99,43\%$ & $92,06\%$ \\*
\nobreakcline{6-8}
& & & & & \textbf{poprawne:} & \multicolumn{2}{c|}{$\mathbf{96,00\%}$} \\
\hline
\multirow{3}{*}{44} & \multirow{3}{*}{Autorzy} & \multirow{3}{*}{$40\%$/$60\%$} & \multirow{3}{*}{$R^n$, wszystko, $d_e$} & \multirow{3}{*}{5} & Byron & $76,42\%$ & $70,43\%$ \\*
 & & & & & Shakespeare & $72,13\%$ & $77,88\%$ \\*
\nobreakcline{6-8}
& & & & & \textbf{poprawne:} & \multicolumn{2}{c|}{$\mathbf{74,12\%}$} \\
\hline
\multirow{3}{*}{45} & \multirow{3}{*}{Autorzy} & \multirow{3}{*}{$40\%$/$60\%$} & \multirow{3}{*}{$R^n$, wszystko, $d_\infty$} & \multirow{3}{*}{5} & Byron & $63,21\%$ & $62,04\%$ \\*
 & & & & & Shakespeare & $66,39\%$ & $67,50\%$ \\*
\nobreakcline{6-8}
& & & & & \textbf{poprawne:} & \multicolumn{2}{c|}{$\mathbf{64,91\%}$} \\
\hline
\multirow{3}{*}{46} & \multirow{3}{*}{Autorzy} & \multirow{3}{*}{$40\%$/$60\%$} & \multirow{3}{*}{$R^n$, wszystko, $d_1$} & \multirow{3}{*}{5} & Byron & $93,40\%$ & $54,10\%$ \\*
 & & & & & Shakespeare & $31,15\%$ & $84,44\%$ \\*
\nobreakcline{6-8}
& & & & & \textbf{poprawne:} & \multicolumn{2}{c|}{$\mathbf{60,09\%}$} \\
\hline
\multirow{3}{*}{47} & \multirow{3}{*}{Autorzy} & \multirow{3}{*}{$40\%$/$60\%$} & \multirow{3}{*}{$R^n$, wszystko, $d_e$} & \multirow{3}{*}{3} & Byron & $76,42\%$ & $71,05\%$ \\*
 & & & & & Shakespeare & $72,95\%$ & $78,07\%$ \\*
\nobreakcline{6-8}
& & & & & \textbf{poprawne:} & \multicolumn{2}{c|}{$\mathbf{74,56\%}$} \\
\hline
\multirow{3}{*}{48} & \multirow{3}{*}{Autorzy} & \multirow{3}{*}{$40\%$/$60\%$} & \multirow{3}{*}{$R^n$, wszystko, $d_e$} & \multirow{3}{*}{11} & Byron & $73,58\%$ & $71,56\%$ \\*
 & & & & & Shakespeare & $74,59\%$ & $76,47\%$ \\*
\nobreakcline{6-8}
& & & & & \textbf{poprawne:} & \multicolumn{2}{c|}{$\mathbf{74,12\%}$} \\
\hline
\multirow{3}{*}{49} & \multirow{3}{*}{Autorzy} & \multirow{3}{*}{$40\%$/$60\%$} & \multirow{3}{*}{$R^n$, wybr. sł., $d_e$} & \multirow{3}{*}{5} & Byron & $32,08\%$ & $58,62\%$ \\*
 & & & & & Shakespeare & $80,33\%$ & $57,65\%$ \\*
\nobreakcline{6-8}
& & & & & \textbf{poprawne:} & \multicolumn{2}{c|}{$\mathbf{57,89\%}$} \\
\hline
\multirow{3}{*}{50} & \multirow{3}{*}{Autorzy} & \multirow{3}{*}{$40\%$/$60\%$} & \multirow{3}{*}{$R^n$, zb. rozm., $d_e$} & \multirow{3}{*}{5} & Byron & $23,58\%$ & $53,19\%$ \\*
 & & & & & Shakespeare & $81,97\%$ & $55,25\%$ \\*
\nobreakcline{6-8}
& & & & & \textbf{poprawne:} & \multicolumn{2}{c|}{$\mathbf{54,82\%}$} \\
\hline
\multirow{3}{*}{51} & \multirow{3}{*}{Autorzy} & \multirow{3}{*}{$40\%$/$60\%$} & \multirow{3}{*}{Jaccard} & \multirow{3}{*}{5} & Byron & $53,77\%$ & $79,17\%$ \\*
 & & & & & Shakespeare & $87,70\%$ & $68,59\%$ \\*
\nobreakcline{6-8}
& & & & & \textbf{poprawne:} & \multicolumn{2}{c|}{$\mathbf{71,93\%}$} \\
\hline
\multirow{3}{*}{52} & \multirow{3}{*}{Autorzy} & \multirow{3}{*}{$40\%$/$60\%$} & \multirow{3}{*}{Wg. sł. kluczowych} & \multirow{3}{*}{5} & Byron & $30,19\%$ & $50,79\%$ \\*
 & & & & & Shakespeare & $74,59\%$ & $55,15\%$ \\*
\nobreakcline{6-8}
& & & & & \textbf{poprawne:} & \multicolumn{2}{c|}{$\mathbf{53,95\%}$} \\
\hline
\multirow{3}{*}{53} & \multirow{3}{*}{Autorzy} & \multirow{3}{*}{$40\%$/$60\%$} & \multirow{3}{*}{N-gramy} & \multirow{3}{*}{5} & Byron & $1,89\%$ & $100,00\%$ \\*
 & & & & & Shakespeare & $100,00\%$ & $53,98\%$ \\*
\nobreakcline{6-8}
& & & & & \textbf{poprawne:} & \multicolumn{2}{c|}{$\mathbf{54,39\%}$} \\
\hline
}\endgroup
}


\section{Dyskusja}
Powyższe wyniki wskazują, że zwiększając parametr \(k\) klasyfikatora w podanych granicach polepsza się jakość klasyfikacji. Najlepsza dla zbioru notatek prasowych 
okazała się nasza autorska metoda bazująca na zbiorach rozmytych. Dla zbioru tekstów znanych pisarzy bardzo dobre wyniki uzyskała metoda naiwnej ekstrakcji z metyryką euklidesową,
a także metoda bazująca na mierze Jaccarda. Niektóre metody nie są skuteczne, gdyż ich czas obliczeń jest zbyt długi dla dużego zbioru tekstów. Do tych metod należą:
metoda n-gramów, naiwna ekstrakcja bazująca na wszystkich słowach a także podobieństwo bazujące na słowach kluczowych. 
{\color{blue}
Sekcja ta powinna zawierać dokładną interpretację uzyskanych wyników
eksperymentów wraz ze szczegółowymi wnioskami z nich płynącymi. Najcenniejsze
są, rzecz jasna, wnioski o charakterze uniwersalnym, które mogą być istotne
przy innych, podobnych zadaniach. Należy również omówić i wyjaśnić wszystkie
napotakane problemy (jeśli takie były). Każdy wniosek powinien mieć poparcie
we wcześniej przeprowadzonych eksperymentach (odwołania do konkretnych
wyników). Jest to jedna z najważniejszych sekcji tego sprawozdania, gdyż
prezentuje poziom zrozumienia badanego problemu.}
\section{Wnioski}
{\color{blue}W tej, przedostatniej, sekcji należy zamieścić podsumowanie
najważniejszych wniosków z sekcji poprzedniej. Najlepiej jest je po prostu
wypunktować. Znów, tak jak poprzednio, najistotniejsze są wnioski o
charakterze uniwersalnym.}


\begin{thebibliography}{2}
\bibitem{1} Niewiadomski A., 2009, \textit{Materiały, przykłady i ćwiczenia do przedmiotu Komputerowe Systemy Rozpoznawania}, skrypt
\bibitem{2} Shakespeare W., 1564-1616, \textit{The Complete Works of William Shakespeare}, Project Gutenberg, \url{http://www.gutenberg.org/ebooks/100}
\bibitem{3} Lord Byron, G. G. 1788-1824, \textit{The Works of Lord Byron, Vol. 4}, Project Gutenberg, \url{http://www.gutenberg.org/ebooks/20158}
\end{thebibliography}
\end{document}
