%\usepackage{tcolorbox}
\usepackage{ucs}
\usepackage[utf8x]{inputenc}
\usepackage[T1]{fontenc}
\usepackage{cite}
%\usepackage{apacite}
\usepackage{cancel}
\usepackage{graphicx}
\usepackage{dcolumn}
\usepackage{color}
\usepackage{colortbl}
\usepackage{enumerate}
\usepackage{url}
\usepackage[squaren]{SIunits}
\usepackage{icomma}
\usepackage{hyperref}
%\usepackage{url}
\usepackage{float}
\usepackage{indentfirst}
\usepackage{amssymb}
\usepackage[intlimits]{amsmath}
\usepackage{amsfonts}
\usepackage{verbatim}
\usepackage{tabulary}
\usepackage{longtable}
\usepackage{rotating}
\usepackage{formular}
\usepackage{marginnote}
\usepackage{listingsutf8}
\usepackage{ifpdf}
%\usepackage{natbib}
\usepackage{fancybox}
\lstset{inputencoding=utf8/latin2,breaklines=true}
\newFRMfield{wyraz}{2cm}
\newif\ifShowAnswers
%\ShowAnswerstrue
\newcommand{\fillin}[2]{\ifShowAnswers{#1}\else{\marginnote{#2}\useFRMfield{wyraz}}\fi}
\newtheorem{definitionx}{Definicja}
\newenvironment{definition}{\begin{definitionx}\normalfont}{\end{definitionx}}
\ifpdf
\newcommand{\obraz}[3]{
\begin{figure}[H]
\centering
\includegraphics[width=9cm]{#1.pdf}
\caption{#3}
\label{fig:#2}
\end{figure}
}
\newcommand{\malyobraz}[3]{
\begin{figure}[H]
\centering
\includegraphics[width=4cm]{#1.pdf}
\caption{#3}
\label{fig:#2}
\end{figure}
}
\newcommand{\podwojnyobraz}[4]{
\begin{figure}[H]
\centering
\includegraphics[width=4cm]{#1.pdf}
\includegraphics[width=4cm]{#2.pdf}
\caption{#4}
\label{fig:#3}
\end{figure}
}
\newcommand{\poziomyobraz}[2]{
\begin{sidewaysfigure}
\centering
\includegraphics[width=25cm]{#1.pdf}
%\caption{#2}
\label{fig:#2}
\end{sidewaysfigure}
}
\else
\newcommand{\obraz}[3]{
\begin{figure}[H]
\centering
\includegraphics[width=9cm]{#1.eps}
\caption{#3}
\label{fig:#2}
\end{figure}
}
\newcommand{\malyobraz}[3]{
\begin{figure}[H]
\centering
\includegraphics[width=4cm]{#1.eps}
\caption{#3}
\label{fig:#2}
\end{figure}
}
\newcommand{\podwojnyobraz}[4]{
\begin{figure}[H]
\centering
\includegraphics[width=4cm]{#1.eps}
\includegraphics[width=4cm]{#2.eps}
\caption{#4}
\label{fig:#3}
\end{figure}
}
\newcommand{\poziomyobraz}[2]{
\begin{sidewaysfigure}
\centering
\includegraphics[width=25cm]{#1.eps}
%\caption{#2}
\label{fig:#2}
\end{sidewaysfigure}
}
\fi
\newcommand{\rysunek}[4]{
\begin{figure}[H]
\centering
\begin{picture}#3
#4
\end{picture}
\caption{#2}
\label{fig:#1}
\end{figure}
}
\newcommand{\tabela}[4]{
\begin{table}[H]
\centering
\begin{tabular}[t]{#3}
#4
\end{tabular}
\caption{#2}
\label{tab:#1}
\end{table}
}

\newcommand{\tabelap}[4]{
\begin{table}[p]
\centering
\begin{tabular}[t]{#3}
#4
\end{tabular}
\caption{#2}
\label{tab:#1}
\end{table}
}

\newcommand{\tabelah}[4]{
\begin{table}[ht]
\centering
\begin{tabular}[t]{#3}
#4
\end{tabular}
\caption{#2}
\label{tab:#1}
\end{table}
}

\newcommand{\tabelat}[4]{
\begin{table}[t]
\centering
\begin{tabular}[t]{#3}
#4
\end{tabular}
\caption{#2}
\label{tab:#1}
\end{table}
}

\newcommand{\tabelab}[4]{
\begin{table}[b]
\centering
\begin{tabular}[t]{#3}
#4
\end{tabular}
\caption{#2}
\label{tab:#1}
\end{table}
}

\newcommand{\ltabela}[3]{
\clearpage
\begin{longtable}[c]{#2}
\caption{#1}\\
#3
\end{longtable}
}

\lstdefinelanguage{Smalltalk}{
  morekeywords={true,false,self,super,nil},
  sensitive=true,
  morecomment=[s]{"}{"},
  morestring=[d]',
}
\lstdefinestyle{SmalltalkStyle}{
  literate={:=}{{$\gets$}}1{^}{{$\uparrow$}}1
} 
